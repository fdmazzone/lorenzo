\chapter{Medida de Lebesgue-Stieltjes}



 \normalmarginpar
En este trabajo denotaremos por $\rr$ al conjunto de todos los números reales, y por $\rr^n$ al conjunto de todas las $n$-uplas con componentes reales.  Si $x\in \rr^n$ escribimos la norma euclídea como $|x|=\sqrt{x_1^2+\cdots x_n^2}$\index[Simbolo]{$\vert x\vert$}, otra norma \deleted{euclídea} que usaremos \replaced{será}{sera} la del \added{\emph{caminante}} $|x|_1=|x_1|+\cdots +|x_n|$. \index[Simbolo]{$\vert x\vert_1 $}  \highlight[comment={repetido}]{Usaremos} el producto interno usual en $\rr^n $ definido por  $x\cdot y=x_1y_1 + \cdots+
x_ny_n$. Se pude pensar al vector $x$ como una matriz de $n\times 1$, y notaremos $x^*$ \index[Simbolo]{$x^*$}a la matriz traspuesta, $1\times n$ de $x$.

El conjunto de todas las funciones continuas, $f:[0,T]\to \rr^n$ será denotado como $C([0,T],\rr^n)$ o simplemente $C([0,T])$\index[Simbolo]{$C([0,T])$} si su codominio es $\rr$.    Escribiremos $f'(x)$ para la derivada de funciones escalares como para funciones vectoriales. \todo{Me parece que hay que decirlo con más precisión, qué tipo de función vectorial estás considerando? es la matriź de derivadas?}

\replaced{Sea $\mathcal{X}$ un espacio topológico, }{Sea $\mathcal{X}$ un espacio topológico} llamaremos $\sigma$-álgebra  de Borel\index{$\sigma$-álgebra} a la $\sigma$-álgebra generada por los conjuntos abiertos de $\mathcal{X}$, y la  denotaremos $\mathscr{B}(\mathcal{X})$. \index[Simbolo]{$\mathscr{B}(\mathcal{X})$}
En particular $\mathscr{B}(\rr)$ \index[Simbolo]{$\mathscr{B}(\rr)$} \replaced{será}{sera} la $\sigma$-álgebra generada por los intervalos abiertos de $\rr$ y $\mathscr{B}([a,b])$\index[Simbolo]{$\mathscr{B}([a,b])$} la generada por los intervalos abiertos \replaced{relativos}{relativo} de $[a,b]$. Diremos que una medida es de Borel si esta definida sobre $\sigma$-álgebra de Borel.\index{$\sigma$-álgebra!Borel}

\reversemarginpar\setlength{\marginparwidth}{2cm}
\deleted[comment={\scriptsize me parece que queda como fuera de lugar aca}]{Sea $A$ y $B$ dos conjuntos, notaremos $A\setminus B$\index[Simbolo]{$A\setminus B$} al complemento de $B$ respecto de $A$.}




\section{Funciones de variación acotada}



\begin{defi}
	Dada $u:[0,T]\to \rr$, definimos la variación \index{variación!} de $u$ como \index[Simbolo]{$\V(u,[0,T])$}
	$$\V(u,[0,T])=\sup_{\substack{ t_i\in [0,T]\\
			0\leq t_1 < t_2< \dots < t_n  \leq T}}\left\lbrace \sum_{i=1}^{n}|u(t_{i+1})-u(t_i)|\right\rbrace $$
\end{defi}

\reversemarginpar
\begin{defi}
	Una función $u:[0,T]\to \rr$ es de variación acotada\index{variación!acotada} en el intervalo $[0,T]$ si $\V(u,[0,T])<\infty$. \\ \todo{no pondría punto y aparte, y de ponerlo tendría que ir una sangría como en todos los párrafos nuevos}
	El conjunto de todas las funciones de variación acotada en $[0,T]$ se denota como $BV([0,T],\rr)$.\index[Simbolo]{$BV([0,T],\rr)$} 
	Si además $u(0)=u(T)$ entonces diremos que $u\in BV_T([0,T],\rr)$. \index[Simbolo]{$BV_T([0,T],\rr)$}
\end{defi}
\begin{thm}\label{T-VB}
	 $u\in BV([0,T],\rr)$ si y sólo si $u$ se puede escribir como la diferencia de dos funciones crecientes en $[0,T]$.
\end{thm}

La demostración de este teorema o más información sobre funciones de variación acotada puede encontrarse en 
 \cite[\deleted{Teorema  2.7.2}]{Carter} %(Teorema  2.7.2). 
\begin{cor}
	Si $u\in BV([0,T],\rr)$, entonces 
	\begin{itemize}
		\item para $0<t<T$ los siguientes límites existen y son finitos
		$$u(t^+)=\lim\limits_{s\to t^+}u(s) \ \ \ \ u(t^-)=\lim\limits_{s\to t^-}u(s);$$\index[Simbolo]{$u(t^+)$}
		\item $u(0^+)$ y $u(T^-)$ existen y son finitos.
	\end{itemize}
\end{cor}



\todo{Faltaría un hilo conductor en esta sección, son como definiciosnes seguidas de teoremas si un nexo entre si}


\section{Medida de Lebesgue-Stieltjes}

El siguiente resultado \replaced{está demostrado}{esta demostrados} en \cite[Proposición 1.2]{folland} y en \cite[Proposición 1.7]{folland}, y muestra que la $\sigma$-álgebra de Borel \replaced{está}{esta} generada por intervalos de la forma $[a,b)$.

\begin{prop}\label{prop:algebra de borel}Sea $\mathcal{A}$ \index[Simbolo]{$\mathcal{A}$}la colección de todos los conjuntos que se escriben como unión finita de intervalos disjuntos de la forma $[a,b)$ con $-\infty<a<b\leq\infty$, más el conjunto $\emptyset$.
	\begin{itemize}
		\item $\mathcal{A}$ es un álgebra.\index{álgebra}
		\item La $\sigma$-álgebra generada por $\mathcal{A}$ es la $\sigma$-álgebra de Borel, ($\mathscr{B}(\rr)$).\index{$\sigma$-álgebra}\index[Simbolo]{$\mathscr{B}(\rr)$}
	\end{itemize}
\end{prop}

Sea $u:\rr \to \rr$  creciente y continua a izquierda y sea $[a_j,b_j)$ intervalos disjuntos para $j=1, \cdots ,m$. Por la \cite[Proposición 1.15]{folland},  $\mu_{0}$ definida como 

$$\mu_{0}\left( \bigcup_{j=1}^m[a_j,b_j)\right)  =\sum_{j=1}^{m}\left(u(b_j)-u(a_j)\right) $$\index[Simbolo]{$\mu_{0}$}
es una premedida \index{premedida} en $\mathcal{A}$ y a partir de una premedida podemos definir  medida exterior en $\mathscr{B}(\rr)$\index{medida! exterior} de la siguiente manera 
$$\mu^{*}(E)=\inf\left\lbrace \sum_{ j=1 }^{\infty}\mu_{0}(A_j) \mid \ A_j\in \mathcal{A}, \ \  E\subset\bigcup_{j=1}^{\infty}A_j \right\rbrace. $$\index[Simbolo]{$\mu^{*}$}
La demostración se puede ver en \cite[Proposición 1.13]{folland}.
%A partir de la medida exterior podemos definir los conjuntos $\mu$-medibles.
\begin{defi}
	Si $u:\rr \to \rr$ es creciente y continua a izquierda, llamaremos medida de Lebesgue-Stieltjes $\mu_{u}$ a la medida inducida por $\mu_{0}$. \index{medida! Lebesgue-Stieltjes}
\end{defi}

El siguiente teorema nos permite escribir a cualquier medida finita sobre la $\sigma$-álgebra de Borel, como una medida de Lebesgue-Stieltjes  . La demostración es análoga a  \cite[Teorema  1.16]{folland}.

\begin{thm}\label{medidas}
	Si $u:\rr \to\rr$ es creciente y continua a izquierda, existe una única medida de Borel $\mu_{u}$ tal que $\mu_{u}([a,b))=u(b)-u(a)$\index[Simbolo]{$\mu_u$} para todo $a$,$b$. Si $G$ es otra función creciente y continua a izquierda, entonces  $\mu_{u}=\mu_{G}$ si y sólo si $u-G$ es constante. Inversamente, si $\mu$ es una medida de Borel  en $\rr$ finita sobre cualquier conjunto acotado de Borel sea
	$$F(x)= \left\{ \begin{array}{lcc}
		\mu([0,x)) &   si  & x > 0 \\
		0 &   si  & x = 0 \\
		-\mu([0,-x)) &   si  & x < 0. 
	\end{array}
	\right. $$
	$F$ es creciente y continua izquierda, y $\mu=\mu_{F}$
\end{thm}  




%
%\begin{defi}
%Sea $u:\rr\to \rr$ una función monótona no decreciente y $[a,b)\subset\rr$ definimos la $u$-medida de $[a,b)$ como \index{medida de Lebesgue-Stieltjes} $\mu_{u}$:
%$$\mu_{u}([a,b)) =\lim\limits_{x\to b^-}f(x) - \lim\limits_{x\to a^+}f(x)=f(b^-)-f(a^+).$$
%\end{defi}

\begin{obs}
	\begin{itemize} 
		\item Si $u(x)=x$ entonces la medida $\mu_{u}$ no es otra que la medida de Lebesgue.
		\item Si $u$ es continua   en $x$, entonces usando  \cite[Teorema 3.28]{Zo} tenemos que 		
		\begin{equation*}
            \begin{split}
			\mu_{u}(\{x\})&=\mu_{u}\left( \bigcap_{n=1}^{\infty}[x,x+1/n)\right) =\lim_{n \to \infty}\mu_{u}[x,x+1/n)
			\\ &=\lim_{n \to \infty}u(x+\frac{1}{n})-u(x)=u(x)-u(x)=0
            \end{split}
		\end{equation*}
		%$$\mu_{u}(\{a\})=u(a)-u(a)=0$$		
		
		\item  Si $u$ es continua  en $b$ entonces 
		$$\mu_{u}([a,b])=\mu_{u}\left([a,b) \bigcup\{b\}\right)=\mu_{u}([a,b))=u(b)-u(a)$$		
	\end{itemize}
\end{obs}

\subsubsection{Medida de  Lebesgue-Stieltjes en un dominio acotado}
 Como $\mathscr{B}([a,b])$ \index[Simbolo]{$\mathscr{B}([a,b])$} es la $\sigma$-álgebra de Borel restringida al intervalo $[a,b]$,  por \ref{prop:algebra de borel} está generada por la familia de intervalos $\mathcal{E}=\left\lbrace [a,x) \mid x\leq b\right\rbrace $.\index[Simbolo]{$\mathcal{E}$}.
 
Sea $u:[a,b]\to\rr$ una función  monótona y continua a izquierda, podemos extender su dominio a todos los números reales, de la siguiente forma:
$$\overline{u}(t)=\left\lbrace \begin{array}{rll}
	u(a) &si & t<a\\
	u(t) & si & a\leq t < b\\
	u(b)& si & t\geq b 
\end{array}\right. $$ \index[Simbolo]{$\overline{u}$}

\begin{obs}
	\begin{itemize}
		\item $\overline{u}$ es monótona, continua a izquierda y en $b$ es continua.
		\item La medida de Lebesgue-Stieltjes generada por $\overline{u}$ cumple que 
		\begin{itemize}
			\item $\mu_{\overline{u}}(\{b\})=0,$
			\item $\mu_{\overline{u}}([a,b])=\mu_{\overline{u}}([a,b)).$
		\end{itemize}
	\end{itemize}
\end{obs}
\begin{defi}
	Sea $u:[a,b]\to\rr$ una función  monótona y continua a izquierda,  definimos la medida de Lebesgue-Stieltjes $\mu_u$ de la siguiente manera,  para $A\subset[a,b]$
	$$\mu_{u}(A)=\mu_{\overline{u}}(A)=\inf\left\lbrace \sum_{ j=1 }^n[\overline{u}(b_j)-\overline{u}(a_j)] \mid A\subset \bigcup_{j=1}^n[a_j,b_j)\right\rbrace. $$
	
	
\end{defi}

\begin{obs}
	\begin{itemize}
        \item $\mu_{u}([a,b])=\mu_{\overline{u}}([a,b])=\overline{u}(b)-\overline{u}(a)=u(b)-u(a),$
		\item Sea $[s,t)\subset[a,b]$ entonces $$\mu_{u}([s,t))=\mu_{\overline{u}}([s,t))=\overline{u}(t)-\overline{u}(s)=u(t)-u(s),$$
		
		\item Sea$[s,t)\supset[a,b]$ entonces
		$$\mu_{u}([s,t))=\mu_{\overline{u}}([s,t))=\overline{u}(t)-\overline{u}(s)=u(b)-u(a).$$
	\end{itemize}
\end{obs}








Ahora podemos definir la medida de Lebesgue-Stieltjes generada por una función de variación acotada.
\begin{thm} \label{Thm:medidas}
	Sea $u\in BV([a,b],\rr)$ y continua a izquierda, existe una medida de Borel $\mu_{u}$ tal que para todo intervalo $[a,b)$ vale que $\mu_{u}([a,b))=u(b)-u(a)$. Además para cualquier función $\mu_{u}$-integrable, $f$ y cualquier conjunto de Borel $A$ notaremos
	$$\int_{A}f(s)\;d\mu_{u}(s)=\int_{A}f(s)\;du(s).$$  \index[Simbolo]{$du$}
\end{thm}
\begin{proof}[Dem.]
Por el teorema \ref{T-VB}, si $u\in BV([a,b],\rr)$ y es continua a izquierda entonces existen $u_1$ y $u_2$ funciones monótonas y continuas a izquierda tal que $u=u_1-u_2$. Luego por el teorema \ref{medidas} las medidas generadas por $u_1$ y $u_2$ son medidas de Borel. Luego si llamamos  $\mu_u=\mu_{u_1}-\mu_{u_2}$  entonces
$$\mu_{u}([a,b))=u(b)-u(a).$$
\end{proof}


\section{Teorema de Lebesgue-Radon-Nikodyn}
Las siguientes definiciones y resultados están basados en \cite[Capitulo 3]{folland}.
%\begin{defi}
	%Sea $\mu:\mathscr{B}(I)\to \rr$ una medida finita llamaremos $\mu^+$
%\end{defi}
\begin{defi}
	Diremos que la medida $\nu$ es absolutamente continua \index{medida! absolutamente continua} respecto de la medida $\mu$, y lo notaremos $\nu\ll\mu$\index[Simbolo]{$\ll$}, si $\nu(A)=0$ cada vez que  $\mu(A)=0$.
\end{defi}
\begin{defi}
	Diremos que dos medidas $\mu$ y $\eta$ son mutuamente singulares \index{medida! mutuamente singulares}, y lo notaremos $\mu\perp \eta$,  si existen conjuntos $E,F\in\mathscr{B}([0,T])$ disjuntos tal que $E\bigcup F=[0,T]$, $\mu(E)=0$ y $\eta(F)=0$. \index[Simbolo]{$\perp$}
\end{defi}
\begin{defi}
	Diremos que una medida $\mu:\mathscr{B}(I)\to \rr$ es continua \index{medida! continua} si para todo $t\in I$ $\mu(\{t\})=0$
\end{defi}
\begin{obs}
	Si $h$ es continua entonces la medida $\mu_{h}$ es continua. Pues $\forall t\in[0,T]$
 \begin{equation*}
	\begin{split}
		\mu_{h}(\{t\})&= \mu_{h}\left( \bigcap_{n=1}^{\infty}[t,t+1/n)\right)=\lim_{n\to \infty}\mu_{h}\left( [t,t+1/n)\right)\\&=\lim_{n\to \infty}h(t+1/n)-h(t)=h(t^+)-h(t)=0.
	\end{split}
 \end{equation*}
\end{obs}

\begin{defi}
	Diremos, que un conjunto $A$ está compactamente incluido \index{compactamente incluido} en el conjunto $B$, si y sólo si $\overline{A}$ es compacto y $\overline{A}\subset B^\circ$. Lo notaremos como $A\Subset B$.
	\index[Simbolo]{$\Subset$}
 \end{defi}
\begin{thm}
    Sea $\mu$ una medida con signo, existen $\mu^-$ y $\mu^+$ medidas positivas tal que $\mu=\mu^+-\mu^-$, y además $\mu^+\perp \mu^-$.\index[Simbolo]{$\mu^+$}\index[Simbolo]{$\mu^-$}
\end{thm}
El teorema anterior se denomina \textsc{descomposición de Jordan}\index{descomposición de Jordan} ver \cite[Capitulo 3.1]{folland} , a las medidas $\mu^+$ y $\mu^-$ se llaman variación positiva\index{variación positiva} y negativa\index{variación negativa} de $\mu$ respectivamente.

\begin{defi}
    Sea $\mu$ una medida con signo, definimos la variación total de $\mu$, como 
    \begin{equation*}
        |\mu|=\mu^+ +\mu^-
    \end{equation*}\index[Simbolo]{$\vert\mu\vert$}
    
\end{defi}
\begin{obs}
 Sea $\mu$ una medida con signo, entonces valen la siguientes propiedades: \label{obs:medida}
 \begin{itemize}
     \item Para cualquier conjunto $E$ $\mu$-medible
 \begin{equation}
     |\mu|(E)=\sup\left\{\sum_{i=1}^n|\mu(E_i)| \mid \bigcup_{i=1}^nE_i=E, \; E_i \text{ disjuntos}  \right\}
 \end{equation}
 
 \item Para cualquier función $f\in L^1(\mu)$ y $E$ un conjunto $\mu$-medible
 $$\left|\int_Ef\;d\mu\right|\leq \int_E|f|\; d|\mu|$$
 \item $ \mu^{\pm}\ll |\mu|$
 \end{itemize}
\end{obs}

\begin{defi} \index[Simbolo]{$L^{p}([0,T],\mu)$}
	Sean $\mu$ una medida de Borel positiva y  $u:[0,T]\to \rr^n$ una función $\mu$-medible. Diremos que:
	\begin{enumerate}
		\item [a)] $u\in L^p([0,T],\mu)$   con $1\leq p<\infty$ si 
		\index[Simbolo]{$\lVert u\rVert_{L^p(\mu)}$} $$\left\| u\right\|_{L^p(\mu)} =\left[ \int_{[0,T)}|u(t)|^p d\mu\right] ^{1/p}<\infty$$
		En caso de que el dominio de $u$ este sobreentendido se puede denotar simplemente como $u\in L^p(\mu)$ o caso de que $\mu$ sea la medida de Lebesgue se denotara simplemente $L^p([0,T])$.
		\item [b)] Sa $\mu$ una medida de Borel, $u\in L^\infty([0,T],\mu)$ \index[Simbolo]{$L^\infty([0,T],\mu)$}si \index[Simbolo]{$\lVert u \rVert_{L^\infty(\mu)}$}
		$$\left\| u\right\|_{L^\infty(\mu)}=\inf_{M}\{|u(t)|<M, \text{para } \mu \text{- c.t.p.}\}  <\infty$$
		
	\end{enumerate}
\end{defi}
\begin{defi}
	Sea $\mu$ una medida de Borel con signo, entonces definimos $L^p(\mu)=L^p(|\mu|)$
\end{defi}




\begin{thm}[Radon-Nikodyn]\label{TL-R-N}
	Sea $\rho$ una medida finita con signo y $\mu$ una medida positiva finita en el espacio $\left( [0,T], \mathscr{B}({0,T})\right) $. Entonces existen $\lambda$ y $\nu$ medidas finitas con signo tal que 
	\begin{equation*}
		\lambda\perp\mu, \quad \nu\ll\mu  \quad \text{ y }\quad \rho=\lambda+\nu.
	\end{equation*}
Más aún, existe una función $h:[0,T]\to\rr$ $\mu$-integrable tal que para todo $A\in \mathscr{B}([0,T])$
\begin{equation*}
	\nu(A)=\int_A h(s)\;d\mu(s).
\end{equation*}
\end{thm}
La siguiente proposición es consecuencia del teorema de \ref{TL-R-N} y está demostrados en \cite[Proposición 3.9]{folland}.
\begin{prop}
    \label{ob1}
	Sea $\mu$ una medida de Borel finita, y $f\in L^1(\mu)$. Si   $$\nu(A)=\int_A f\; d\mu$$ entonces para toda función $g\in L^1(\nu)$ vale que $gf\in L^1(\mu)$ y 
	\begin{equation*}
	    \int_A g(s)\;d\nu=\int_Ag(s)f(s)\;d\mu.
	\end{equation*}

\end{prop}



\begin{lem}\label{obs3}
	Supongamos que  $\nu\ll\mu$ y sea $\mu^*$ una medida tal que $\nu(A)\leq \mu^*(A)$, entonces para toda función $\psi\in L^1(\nu)$ vale que 
	$$\int_A h(r)\psi(r)\;d\mu(r)=\int_A\psi(r)\;d\nu(r)\leq \int_A\psi(r)\;d\mu^*,$$
	donde $h$ es la derivada de $\nu$ respecto a la medida $\mu$.
\end{lem}
\begin{proof}[Dem.]
La igualdad es consecuencia del teorema \ref{TL-R-N}. Veamos la desigualdad. Vaosa a tomar el conjunto $A$ como un intervalo abierto $I$ (los cuales generan la $\sigma$-álgebra de Borel).
\begin{itemize}
	\item Si $\psi$ es una función característica, es decir $\psi(t)=X_E(t)$.
	\begin{multline*}
		\int_I \psi(r) \;d\nu(r)=\int_{I\bigcap E} \;d\nu(r)=\nu \left(I\bigcap E\right)\leq \mu^*\left(I\bigcap E\right)\\
		\leq\int_{I\bigcap E} \;d\mu^*(r)=\int_I\psi(r) \;d\mu^*(r).
	\end{multline*}
\item Si $\psi$ es una función simple, dada por $\psi(t)=\displaystyle\sum_{i=1}^{m}a_iX_{E_i}(t)$ entonces
\begin{equation*}
\begin{split}
	\int_I \psi(r)\; d\nu(r)&=\sum_{i=1}^{m}\left( a_i\int_IX_{E_i}(r) \;d\nu(r)\right) \\ &\leq
	\sum_{i=1}^{m}a_i\left( \int_IX_{E_i}(r) \;d\mu^*(r)\right) =\int_I \psi(r)\; d\mu^*(r).
 \end{split}
\end{equation*}
\item Si $\psi$ es una función integrable positiva, entonces existe una sucesión de funciones simples $f_1(t)\leq f_2(t)\leq\cdots$ tal que $\lim\limits_{k\to\infty}f_k(t)=\psi(t)$. Luego, usando Beppo-Levi,
\begin{equation*}
\begin{split}
	\int_I\psi(r)\;d\nu(r)&=\lim\limits_{k\to\infty}\int_I f_k(r)\;d\nu(r)\\
	&\leq \lim\limits_{k\to\infty}\int_I f_k(r)\;d\mu^*(r)=\int_I\psi(r)\;d\mu^*(r).
 \end{split}
\end{equation*}
	\item Si $\psi$ es cualquier función integrable, entonces se puede descomponer como diferencia de dos funciones positivas, es decir $\psi=\psi^+-\psi^-$  y por lo anterior satisface  que 
$$\int_I\psi(r)\;d\nu(r)\leq \int_I\psi(r)\;d\mu^*.$$	
\end{itemize}
Luego, si se verifica para cualquier intervalo $I$ entonces como generan la $\sigma$-álgebra de Borel, se satisface para cualquier conjunto $A$ de Borel.
\end{proof}








\begin{defi}
	Sea $\mu$ una medida, llamaremos $D_{\mu}$ o simplemente $D$, al conjunto de los puntos de discontinuidad \index{punto de discontinidad} de $\mu$. Es decir, 
	$$D=\{t \in [0,T]  \mid  \mu(\{t\})\neq 0\}.$$\index[Simbolo]{$D$}
\end{defi}

\begin{lem}\label{D numerable}
	Si $\mu:\mathscr{B}([0,T])\to \rr$ es una medida finita y positiva, entonces el conjunto $D$ es numerable.
\end{lem}
\begin{proof}[Dem.]
	Para $n\in\nn$ definimos el conjunto $D_n=\{\tau|\mu(\{\tau\})>1/n\}$, entonces $D=\bigcup_{n=1}^\infty D_n$. El conjunto $D_n$ es finito, porque de lo contrario va a existir una sucesión de elementos $\{a_k\}_{k=1}^\infty\subset D_n$ tal que 
	\begin{equation*}
		\mu([0,T])\geq\mu(D_n)\geq\sum_{k=1}^{\infty}\mu(\{a_k\})>\sum_{k=1}^{\infty}\dfrac{1}{n}=\infty,
	\end{equation*}
	lo cual es un absurdo pues $\mu$ es una medida finita. Por lo tanto, $D_n$ es finito y $D$ es la unión numerable de conjuntos finitos. 
\end{proof}
