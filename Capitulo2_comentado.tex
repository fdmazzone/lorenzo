\chapter{Medida de Lebesgue-Stieltjes}




En este trabajo denotaremos por $\rr$ al conjunto de todos los números reales y por $\rr^n$ al conjunto de todas las $n$-uplas con componentes reales.  Si $x\in \rr^n$, escribiremos la norma euclídea como $|x|=\sqrt{x_1^2+\cdots x_n^2}$\index[Simbolo]{$\vert x\vert$}, y otra norma que usaremos será la del caminante  $|x|_1=|x_1|+\cdots +|x_n|$. \index[Simbolo]{$\vert x\vert_1 $}  
El producto interno que usaremos es el  usual en $\rr^n $, definido por  $x\cdot y=x_1y_1 + \cdots+
x_ny_n$. Se puede pensar al vector $x$ como una matriz de $n\times 1$, y notaremos con $x^*$ \index[Simbolo]{$x^*$}a la matriz traspuesta $1\times n$ de $x$.

\textcolor{blue}{El conjunto de todas las funciones continuas, $f:[0,T]\to \rr^n$ será denotado como $C([0,T],\rr^n)$ o simplemente $C([0,T])$\index[Simbolo]{$C([0,T])$} si su codominio es $\rr$.    Escribiremos $f'(x)$ para la derivada de funciones escalares como para funciones de $\rr$ a $\rr^n$. }

Sea $\mathcal{X}$ un espacio topológico. Se define \added{la }  $\sigma$-álgebra  de Borel\index{$\sigma$-álgebra} \replaced{como }{a} la $\sigma$-álgebra generada por los conjuntos abiertos de $\mathcal{X}$, y la  denotaremos $\mathscr{B}(\mathcal{X})$. \index[Simbolo]{$\mathscr{B}(\mathcal{X})$}
En particular, $\mathscr{B}(\rr)$ \index[Simbolo]{$\mathscr{B}(\rr)$}será la $\sigma$-álgebra generada por los intervalos abiertos de $\rr$ y $\mathscr{B}([a,b])$\index[Simbolo]{$\mathscr{B}([a,b])$} la generada por los intervalos abiertos relativos de $[a,b]$. Diremos que una medida es de Borel si esta definida sobre \added{la } $\sigma$-álgebra de Borel.\index{$\sigma$-álgebra!Borel} \comment{Pondría en otro lado esta notación de diferencia de conjuntos} Dados dos conjuntos $A$ y $B$, notaremos $A\setminus B$\index[Simbolo]{$A\setminus B$} al complemento de $B$ respecto de $A$.




\section{Funciones de variación acotada}
\textcolor{blue}{Cuando definimos \ref{eq:problema A} dijimos que $\varphi$ era una función de variación acotada y que entendíamos a  $d\varphi$ como la medida de Lebesgue-Stieltjes de $\varphi$. En esta sección daremos las definiciones y principales resultados, que nos permitan definir la medida de Lebesgue-Stieltjes de una función de variación acotada.}

\begin{defi}
	Dada $u:[0,T]\to \rr$\comment{En esta y las definiciones de abajo, no vamos a necesitar funciones con valores a $\rr^n$?}, definimos la variación \index{variación!} de $u$ como \index[Simbolo]{$\V(u,[0,T])$}\label{def:variación}
	$$\V(u,[0,T])=\sup_{\substack{ t_i\in [0,T]\\
			0\leq\reversemarginpar \textrm{\todo{Pondría $0=t_1$ y $t_n=T$ y debajo del supremo solo el símbolo de la partición $P$}}t_1 < t_2< \dots < t_n  \leq T}}\left\lbrace \sum_{i=1}^{n-1}|u(t_{i+1})-u(t_i)|\right\rbrace $$
\added{donde el supremo se toma sobre todas las particiones $P=\{t_1,\ldots,t_n\}$ del intervalo $[0,T]$} 
\end{defi}

\begin{defi}
\textcolor{blue}{ Una función $u:[0,T]\to \rr$ es de variación acotada\index{variación!acotada} en el intervalo $[0,T]$ si $\V(u,[0,T])<\infty$.
	El conjunto de todas las funciones de variación acotada en $[0,T]$ se denota como $BV([0,T],\rr)$.\index[Simbolo]{$BV([0,T],\rr)$} Si además $u(0)=\alpha$ diremos que $u\in BV_\alpha([0,T],\rr^n)$. \index[Simbolo]{$BV_\alpha([0,T],\rr^n)$}}
	%Si además $u(0)=u(T)$ entonces diremos que $u\in BV_T([0,T],\rr)$. \index[Simbolo]{$BV_T([0,T],\rr)$}
\end{defi}
\begin{lem} \label{lem:var-acotada}
    \textcolor{blue}{ $u\in BV_\alpha([0,T],\rr^n)$, entonces para todo $t\in[0,T]$ existe $K>0$ tal que $|u(t)|\leq K$.  }
\end{lem}
\begin{proof}[Dem.]\textcolor{blue}{
Si $u\in BV_\alpha([0,T],\rr^n)$, entonces
    \begin{equation}
        |u(t)|\leq |u(t)-u(0)|+|\alpha|\leq V(u,[0,t])+|\alpha|\leq V(u,[0,T])+|\alpha|.
    \end{equation}
Luego llamando $K=V(u,[0,T])+|\alpha|$, tenemos que $|u(t)|\leq K$.  }
    
\end{proof}


    
\begin{prop}\textcolor{blue}{ \label{pro:completo}
     Si $d(u,v)=V(u-v,[0,T])$ \index[Simbolo]{$d$}, entonces $\left(BV_\alpha([0,T],\rr^n),d\right)$ es un espacio métrico completo.}
\end{prop}
\begin{proof}[Dem.]
    \textcolor{blue}{Para que $\left(BV_\alpha([0,T],\rr^n),d\right)$ sea un espacio métrico la función $d$ tiene que ser una métrica. Sean $u,v,w\in BV_\alpha([0,T]$:
\begin{itemize}
\item Si $d(u,v)=0$ entonces $V(u-v,[0,T])=0$, por lo tanto $v-u$ es constante y como $u(0)=v(0)$ resulta que $u=v$. 
\item $d(u,u)=V(u-u,[0,T])=0$. \comment{Me parece que este  y el inciso que sigue son redundantes, porque el primero de todos era un si y solo si}
\item $d(u,v)=V(u-v,[0,T])>|u(0)-v(0)-u(T)+v(T)|>0$.
\item Usando la definición de variación \ref{def:variación}
\begin{equation*}
    \begin{split}
    d(u,v)&=\sup_{\substack{ t_i\in [0,T]\\
0\leq t_1 < t_2< \dots < t_n  \leq T}}\left\lbrace \sum_{i=1}^{n-1}|u(t_{i+1})-v(t_{i+1})-u(t_i)+v(t_{i})|\right\rbrace\\&=
\sup_{\substack{ t_i\in [0,T]\\
0\leq t_1 < t_2< \dots < t_n  \leq T}}\left\lbrace \sum_{i=1}^{n-1}|v(t_{i+1})-u(t_{i+1})-v(t_i)+u(t_{i})|\right\rbrace\\&=d(v,u).
    \end{split}
\end{equation*}
\item Sea $0=t_1,t_2,\ldots t_k=T$ una partición del intervalo $[0,T]$, entonces
\begin{equation*}
    \begin{split}
        |u(t_{i+1})-v(t_{i+1})-u(t_i)+v(t_{i})|&   \leq |u(t_{i+1})-w(t_{i+1})-u(t_i)+w(t_{i})|\\&+ |w(t_{i+1})-v(t_{i+1})-w(t_i)+v(t_{i})|,
    \end{split}
\end{equation*}
sumando para $i=1$ hasta $k-1$ y tomando supremo sobre todas las particiones del intervalo $[0,T]$ tenemos que 
$$V(u-v,[0,T])\leq V(u-w,[0,T])+V(w-v,[0,T]).$$
Por lo tanto $d(u,v)\leq d(u,w)+d(w,v)$.
    \end{itemize}
Veamos que $\left(BV_\alpha([0,T],\rr^n),d\right)$ es completo. Sea $\{u_n\}$ una sucesión de Cauchy\replaced{. Como toda sucesión de Cauchy es acotada,}{,   luego} existe una constante $K$ tal que $\forall n$ $d(u_n,0)=V(u_n,[0,T])\leq K$. Además por \added{Lema } \ref{lem:var-acotada} para cualquier $t\in[0,T]$
\begin{equation} \label{eq:desigualdad d}
|u_n(t)|\leq|u_n(t)-u_n(0)|+|u_n(0)|\leq V(u_n,[0,T])+|\alpha|\leq K.
\end{equation}
Luego por el \replaced{Teorema}{teorema} de Helly's \replaced{(ver \cite{Natanson})}{\cite{Natanson}}
\comment{Va a quedar mejor si identificas con más presisión el teorema}
existe una \replaced{subsucesión}{subsecesión} convergente\replaced{. Una sucesión de Cauchy que posee una subsucesión convergente es, en si misma, convergente. Por}{, y por} lo tanto la sucesión es convergente.
}
\end{proof}







El siguiente resultado permite caracterizar a las funciones de variación acotada.

\begin{thm}\label{T-VB}
	 $u\in BV([0,T],\rr)$ si y sólo si $u$ se puede escribir como la diferencia de dos funciones crecientes en $[0,T]$.
\end{thm}

La demostración de este teorema y más información sobre funciones de variación acotada puede encontrarse en \cite{Carter}. %(Teorema  2.7.2). 
\begin{cor}
	Si $u\in BV([0,T],\rr)$, entonces 
	\begin{itemize}
		\item para $0<t<T$ los siguientes límites existen y son finitos
		$$u(t^+)=\lim\limits_{s\to t^+}u(s) \ \ \ \ u(t^-)=\lim\limits_{s\to t^-}u(s);$$\index[Simbolo]{$u(t^+)$}
		\item $u(0^+)$ y $u(T^-)$ existen y son finitos.
	\end{itemize}
\end{cor}






\section{Medida de Lebesgue-Stieltjes}
A continuación, vamos a enumerar algunos resultados necesarios para definir la medida de Lebesgue-Stieljes asociada a una función de variación acotada. Para un desarrollo mas detallado ver \cite{folland}.
\subsection{Medida de Lebesgue-Stieltjes en $\mathbb{R}$}
Vamos a definir la medida de Labesgue-Stieltjes para una función \replaced{no decreciente}{de creciente} y continua a izquierda. El mismo desarrollo se puede hacer para funciones \replaced{no crecientes}{decrecientes}, con la salvedad de que la medida generada \replaced{es}{en} negativa.


\begin{defi}\label{def:premedida}
Sea $\mathcal{A}$ un álgebra\added{ de subconjuntos de $X$, es decir $\mathcal{A}$ es cerrado para uniones e intersecciones finitas y $\emptyset,X\in\mathcal{A}$}. Diremos que la función $\mu_0:\mathcal{A}\to [0,\infty)$ es una premedida\index{premedida}, si cumple 
\begin{itemize}
    \item $\mu_0(\emptyset)=0,$
    \item si $\displaystyle\left\{A_i\right\}_1^\infty$ es una familia de conjuntos disjuntos tal que $\displaystyle\bigcup_1^\infty A_i\in \mathcal{A}$, entonces $\displaystyle\mu_0\left(\bigcup_1^\infty A_i\right)\leq \sum_1^\infty \mu_0\left(A_i\right).$ 
\end{itemize}

\end{defi}
\textcolor{blue}{Vamos a notar con $\mathcal{A}$ \index[Simbolo]{$\mathcal{A}$} a la colección de todos los conjuntos que se escriben como unión finita de intervalos disjuntos de la forma $[a,b)$ con $-\infty<a<b<\infty$, más el conjunto $\emptyset$. En proposición 1.2 y  1.7 de \cite{folland} se muestra que  $\mathcal{A}$ es un álgebra y que además genera la $\sigma$-álgebra de Borel.}

%\begin{prop}El conjunto $\mathcal{A}$ cumple que:
%\label{prop:algebra de borel}
%	\begin{itemize}
%		\item $\mathcal{A}$ es un álgebra.\index{álgebra}
%		\item La $\sigma$-álgebra generada por $\mathcal{A}$ es la $\sigma$-álgebra de Borel, ($\mathscr{B}(\rr)$).\index{$\sigma$-álgebra}\index[Simbolo]{$\mathscr{B}(\rr)$}
%	\end{itemize}
%\end{prop}


Sea  $u:\rr \to \rr$  creciente y continua a izquierda y sean  $[a_j,b_j)$ intervalos disjuntos para $j=1, \cdots ,m$. Por la proposición 1.15 de \cite{folland}, la función  $\mu_{0}$ definida como 

$$\mu_{0}\left( \bigcup_{j=1}^m[a_j,b_j)\right)  =\sum_{j=1}^{m}\left(u(b_j)-u(a_j)\right) $$\index[Simbolo]{$\mu_{0}$}
es una premedida  en $\mathcal{A}$. A partir de una premedida podemos definir  medida exterior para cualquier conjunto arbitrario $E\subset\rr$ \index{medida! exterior} de la siguiente manera 
$$\mu^{*}(E)=\inf\left\lbrace \sum_{ j=1 }^{\infty}\mu_{0}(A_j) \mid \ A_j\in \mathcal{A}, \ \  E\subset\bigcup_{j=1}^{\infty}A_j \right\rbrace. $$\index[Simbolo]{$\mu^{*}$}En $\mathscr{B}(\rr)$, $\mu^{*}$ satisface los axiomas de medida \cite[Proposición 1.13]{folland}.
%A partir de la medida exterior podemos definir los conjuntos $\mu$-medibles.
\begin{defi}
	Si $u:\rr \to \rr$ es creciente y continua a izquierda, llamaremos medida de Lebesgue-Stieltjes $\mu_{u}$ a la medida inducida por $\mu_{0}$ mediante el procedimiento anterior. \index{medida! Lebesgue-Stieltjes}
\end{defi}

El siguiente teorema nos permite escribir cualquier medida finita sobre la $\sigma$-álgebra de Borel como una medida de Lebesgue-Stieltjes. La demostración es análoga a  \cite[Teorema  1.16]{folland}.

\begin{thm}\label{medidas}
	Si $u:\rr \to\rr$ es creciente y continua a izquierda, existe una única medida de Borel $\mu_{u}$ tal que $\mu_{u}([a,b))=u(b)-u(a)$\index[Simbolo]{$\mu_u$} para todo $a$,$b$. Si $G$ es otra función creciente y continua a izquierda, entonces  $\mu_{u}=\mu_{G}$ si y sólo si $u-G$ es constante. Recíprocamente, si $\mu$ es una medida de Borel  en $\rr$ finita sobre cualquier conjunto acotado de Borel y se considera
	$$F(x)= \left\{ \begin{array}{lcc}
		\mu([0,x)) &   si  & x > 0 \\
		0 &   si& x = 0 \\
		-\mu([0,-x)) &   si  & x < 0. 
	\end{array}
	\right. $$
	Entonces $F$ es creciente y continua izquierda, y $\mu=\mu_{F}$.
\end{thm}  




%
%\begin{defi}
%Sea $u:\rr\to \rr$ una función monótona no decreciente y $[a,b)\subset\rr$ definimos la $u$-medida de $[a,b)$ como \index{medida de Lebesgue-Stieltjes} $\mu_{u}$:
%$$\mu_{u}([a,b)) =\lim\limits_{x\to b^-}f(x) - \lim\limits_{x\to a^+}f(x)=f(b^-)-f(a^+).$$
%\end{defi}

\begin{obs} \vphantom{a}
	\begin{itemize} 
		\item Si $u(x)=x$, entonces la medida $\mu_{u}$ no es otra que la medida de Lebesgue.
		\item Si $u$ es continua   en $x$, entonces usando  \cite[Teorema 3.28]{Zo} tenemos que 		
		\begin{equation*}
            \begin{split}
			\mu_{u}(\{x\})&=\mu_{u}\left( \bigcap_{n=1}^{\infty}[x,x+1/n)\right) =\lim_{n \to \infty}\mu_{u}\left([x,x+1/n)\right)
			\\ &=\lim_{n \to \infty}\left(x+\frac{1}{n}\right)-u(x)=u(x)-u(x)=0.
            \end{split}
		\end{equation*}
		%$$\mu_{u}(\{a\})=u(a)-u(a)=0$$		
		
		\item  Si $u$ es continua  en $b$, entonces 
		$$\mu_{u}([a,b])=\mu_{u}\left([a,b) \cup\{b\}\right)=\mu_{u}([a,b))=u(b)-u(a).$$		
	\end{itemize}
\end{obs}

\subsection{Medida de  Lebesgue-Stieltjes en un dominio acotado}
\textcolor{blue}{ Por la proposición 1.2 de \cite{folland}, la familia de conjuntos $\mathcal{E}=\{(-\infty,x),\mid x\in\rr\}$ genera la $\sigma-$álgebra de Borel. Llamaremos $\mathscr{B}([a,b])$ \index[Simbolo]{$\mathscr{B}([a,b])$} a la $\sigma$-álgebra de Borel restringida al intervalo $[a,b]$, la cual  está generada por la familia de intervalos $\mathcal{E}\cap [a,b]=\left\lbrace [a,x) \mid x\leq b\right\rbrace $ \index[Simbolo]{$\mathcal{E}$}.}
 
Sea $u:[a,b]\to\rr$ una función  creciente y continua a izquierda, podemos extender su dominio a todos los números reales, de la siguiente forma:
$$\overline{u}(t)=\left\lbrace \begin{array}{rll}
	u(a) &si & t<a\\
	u(t) & si & a\leq t < b\\
	u(b)& si & t\geq b.
\end{array}\right. $$ \index[Simbolo]{$\overline{u}$}

    
\begin{obs} La función $\overline{u}$ tiene las siguientes propiedades:
	\begin{itemize}
		\item $\overline{u}$ es creciente, continua a izquierda y en $b$ es continua.
		\item La medida de Lebesgue-Stieltjes generada por $\overline{u}$ cumple que 
		\begin{enumerate}
			\item[I.] $\mu_{\overline{u}}(\{b\})=0,$
			\item[II.] $\mu_{\overline{u}}([a,b])=\mu_{\overline{u}}([a,b)).$
		\end{enumerate}
	\end{itemize}
\end{obs}
\begin{defi}
	Sea $u:[a,b]\to\rr$ una función  creciente y continua a izquierda. Para $A\subset[a,b]$  definimos la medida de Lebesgue-Stieltjes $\mu_u$ por 
	\begin{equation*}
	\mu_{u}(A)=\mu_{\overline{u}}(A)=\inf\left\lbrace \sum_{ j=1 }^n[\overline{u}(b_j)-\overline{u}(a_j)] \mid A\subset \bigcup_{j=1}^n[a_j,b_j)\right\rbrace. 
 \end{equation*}
\end{defi}

\begin{obs}  \vphantom{a}
	\begin{itemize}
        \item $\mu_{u}([a,b])=\mu_{\overline{u}}([a,b])=\overline{u}(b)-\overline{u}(a)=u(b)-u(a)$.
		\item Sea $[s,t)\subset[a,b]$ entonces $$\mu_{u}([s,t))=\mu_{\overline{u}}([s,t))=\overline{u}(t)-\overline{u}(s)=u(t)-u(s).$$
		
		\item Sea$[s,t)\supset[a,b]$ entonces
		$$\mu_{u}([s,t))=\mu_{\overline{u}}([s,t))=\overline{u}(t)-\overline{u}(s)=u(b)-u(a).$$
	\end{itemize}
\end{obs}







\textcolor{blue}{Si en la definición \ref{def:premedida} tomamos a la función $\mu_0:\mathcal{A}\to \rr$, entonces la medida generada se llama medida con signo. \index{medida con signo}Veamos que para $f\in BV([a,b],\rr^n)$ la medida asociada a $f$, es una medida con signo.}



\begin{thm} \label{Thm:medidas}
    	Sea $u\in BV([a,b],\rr)$ y continua a izquierda. Existe una medida de Borel con signo $\mu_{u}$, tal que $$\mu_{u}([a,b))=u(b)-u(a).$$  Para cualquier función $f$ $\mu_{u}$-integrable  y cualquier conjunto de Borel $A$, notaremos
	$$\int_{A}f(s)\;d\mu_{u}(s)=\int_{A}f(s)\;du(s).$$  \index[Simbolo]{$du$}
\end{thm}





\begin{proof}[Dem.]
Por el teorema \ref{T-VB}, si $u\in BV([a,b],\rr)$ y es continua a izquierda\textcolor{red}{,} entonces existen $u_1$ y $u_2$ funciones crecientes y continuas a izquierda tal que $u=u_1-u_2$. Luego\textcolor{red}{,} por el teorema \ref{medidas}\textcolor{red}{,}  las medidas generadas por $u_1$ y $u_2$ son medidas de Borel. Ahora si llamamos  $\mu_u=\mu_{u_1}-\mu_{u_2}$\textcolor{red}{,}   entonces
$$\mu_{u}([a,b))=u(b)-u(a).$$
\end{proof}


\section{Medida de Lebesgue-Stieljes con signo}

Las siguientes definiciones y resultados están basados en \cite[Capitulo 3]{folland}.

\begin{defi}
	Diremos que la medida con signo $\nu$ es absolutamente continua \index{medida! absolutamente continua} respecto de la medida con signo $\mu$, y  notaremos $\nu\ll\mu$\index[Simbolo]{$\ll$}, si $\nu(A)=0$ cada vez que  $\mu(A)=0$.
\end{defi}


\begin{defi}
	Diremos que dos medidas con signo $\mu$ y $\eta$ son mutuamente singulares\index{medida! mutuamente singulares}, y  notaremos $\mu\perp \eta$,  si existen conjuntos $E,F\in\mathscr{B}([0,T])$ disjuntos tal que $E\cup F=[0,T]$, $\mu(E)=0$ y $\eta(F)=0$. \index[Simbolo]{$\perp$}
\end{defi}

\begin{defi}
	Diremos que una medida con signo $\mu$ es continua \index{medida! continua} si para todo $t\in I$ $\mu(\{t\})=0$
\end{defi}
\begin{obs}\label{obs:medida continua}
	Sea $h$ una  función continua. Entonces la medida con signo $\mu_{h}$ es continua. Pues $\forall t\in[0,T]$
 \begin{equation*}
	\begin{split}
		\mu_{h}(\{t\})&= \mu_{h}\left( \bigcap_{n=1}^{\infty}[t,t+1/n)\right)=\lim_{n\to \infty}\mu_{h}\left( [t,t+1/n)\right)\\&=\lim_{n\to \infty}h(t+1/n)-h(t)=h(t^+)-h(t)=0.
	\end{split}
 \end{equation*}
\end{obs}

\begin{defi}
	Diremos que un conjunto $A$ está compactamente incluido \index{compactamente incluido} en el conjunto $B$, si y sólo si $\overline{A}$ es compacto y $\overline{A}\subset B^\circ$. Lo notaremos como $A\Subset B$.
	\index[Simbolo]{$\Subset$}
 \end{defi}
\begin{thm}
    Sea $\mu$ una medida con signo, entonces existen $\mu^-$ y $\mu^+$ medidas positivas tal que $\mu=\mu^+-\mu^-$, y además $\mu^+\perp \mu^-$.\index[Simbolo]{$\mu^+$}\index[Simbolo]{$\mu^-$}
\end{thm}
El teorema anterior se denomina \textsc{descomposición de Jordan}\index{descomposición de Jordan} ver \cite[Capitulo 3.1]{folland}, y a las medidas $\mu^+$ y $\mu^-$ se las llama variación positiva\index{variación positiva} y negativa\index{variación negativa} de $\mu$, respectivamente.

\begin{defi}
    Sea $\mu$ una medida con signo. Definimos la variación total de $\mu$, como 
    \begin{equation*}
        |\mu|=\mu^+ +\mu^-.
    \end{equation*}\index[Simbolo]{$\vert\mu\vert$}
\end{defi}


\begin{obs}
 Sea $\mu$ una medida con signo, entonces valen las siguientes propiedades: \label{obs:medida}
 \begin{itemize}
 \item $|\mu|$ es una medida de Borel positiva.
     \item Para cualquier conjunto $E$ $\mu$-medible,  se verifica
 \begin{equation}
     |\mu|(E)=\sup\left\{\sum_{i=1}^n|\mu(E_i)| \text{ donde }\bigcup_{i=1}^nE_i=E, \; E_i \text{ disjuntos}  \right\}.
 \end{equation}

 \item Para cualquier función $f\in L^1(\mu)$ y $E$ un conjunto $\mu$-medible, vale que
 $$\left|\int_Ef\;d\mu\right|\leq \int_E|f|\; d|\mu|.$$
 \item $ \mu^{\pm}\ll |\mu|$.
 \end{itemize}
\end{obs}

\begin{defi} \index[Simbolo]{$L^{p}([0,T],\mu)$}
	Sean $\mu$ una medida de Borel positiva  y  $u:[0,T]\to \rr^n$ una función $\mu$-medible. Diremos que:
	\begin{enumerate}
		\item [a)] $u\in L^p([0,T],\mu)$   con $1\leq p<\infty$ si 
		\index[Simbolo]{$\lVert u\rVert_{L^p(\mu)}$} $$\left\| u\right\|_{L^p(\mu)} =\left[ \int_{[0,T)}|u(t)|^p d\mu\right] ^{1/p}<\infty.$$
		En caso de que el dominio de $u$ esté sobreentendido notaremos $u\in L^p(\mu)$ y cuando $\mu$ sea la medida de Lebesgue se denotara simplemente $L^p([0,T])$.
  
		\item [b)] Sea $\mu$ una medida de Borel positiva. Diremos que  $u\in L^\infty([0,T],\mu)$ \index[Simbolo]{$L^\infty([0,T],\mu)$}si \index[Simbolo]{$\lVert u \rVert_{L^\infty(\mu)}$}
		$$\left\| u\right\|_{L^\infty(\mu)}=\inf\{M \mid |u(t)|<M, \text{para } \mu \text{- c.t.p.}\}  <\infty.$$
		
	\end{enumerate}
\end{defi}
\begin{defi}
	Sea $\mu$ una medida de Borel con signo, entonces definimos $L^p(\mu)=L^p(|\mu|)$.
\end{defi}




\begin{thm}[Radon-Nikodyn]\label{TL-R-N}
	Sea $\rho$ una medida finita con signo y $\mu$ una medida positiva finita en el espacio $\left( [0,T], \mathscr{B}([0,T])\right) $. Entonces existen $\lambda$ y $\nu$ medidas finitas con signo tal que 
	\begin{equation*}
		\lambda\perp\mu, \quad \nu\ll\mu  \quad \text{ y }\quad \rho=\lambda+\nu.
	\end{equation*}
Más aún, existe una función $h:[0,T]\to\rr$ $\mu$-integrable tal que para todo $A\in \mathscr{B}([0,T])$
\begin{equation*}
	\nu(A)=\int_A h(s)\;d\mu(s).
\end{equation*}
\end{thm}
\textcolor{blue}{A la función $h$ se la suele llamar la derivada de $\nu$ respecto la medida $\mu$, y se nota  $\dfrac{d\nu}{d\mu}$.
La siguiente proposición es consecuencia del teorema de \ref{TL-R-N} y está demostrado en \cite[Proposición 3.9]{folland}.}




\begin{prop}
    \label{ob1}
	Sean $\mu$ una medida de Borel con signo y finita, y  $f\in L^1(\mu)$. Si   $$\nu(A)=\int_A f\; d\mu,$$ entonces para toda función $g\in L^1(\nu)$ vale que $gf\in L^1(\mu)$ y 
	\begin{equation*}
	    \int_A g(s)\;d\nu=\int_Ag(s)f(s)\;d\mu.
	\end{equation*}

\end{prop}



\begin{lem}\label{obs3}
	Supongamos que  $\nu\ll\mu$ y sea $\mu^*$ una medida tal que $\nu(A)\leq \mu^*(A)$, entonces para toda función $\psi\in L^1(\nu)$ vale que 
	$$\int_A\psi(r)\;d\nu(r)\leq \int_A\psi(r)\;d\mu^*.$$
	
\end{lem}
\begin{proof}[Dem.]
 Vamos a  tomar el conjunto $A$ como un intervalo abierto $I$.
 
\begin{itemize}
	\item Si $\psi$ es una función característica, es decir, $\psi(t)=X_E(t)$. Tenemos
	\begin{multline*}
		\int_I \psi(r) \;d\nu(r)=\int_{I\cap E} \;d\nu(r)=\nu \left(I\cap E\right)\leq \mu^*\left(I\cap E\right)\\
		\leq\int_{I\cap E} \;d\mu^*(r)=\int_I\psi(r) \;d\mu^*(r).
	\end{multline*}
\item Si $\psi$ es una función simple, dada por $\psi(t)=\displaystyle\sum_{i=1}^{m}a_iX_{E_i}(t)$ entonces
\begin{equation*}
\begin{split}
	\int_I \psi(r)\; d\nu(r)&=\sum_{i=1}^{m}\left( a_i\int_IX_{E_i}(r) \;d\nu(r)\right) \\ &\leq
	\sum_{i=1}^{m}a_i\left( \int_IX_{E_i}(r) \;d\mu^*(r)\right) =\int_I \psi(r)\; d\mu^*(r).
 \end{split}
\end{equation*}
\item Si $\psi$ es una función integrable positiva, entonces existe una sucesión de funciones simples $f_1(t)\leq f_2(t)\leq\cdots$ tal que $\lim\limits_{k\to\infty}f_k(t)=\psi(t)$. Luego, usando  el Teorema de Beppo-Levi \cite[Teorema 5.6]{Zo},
\begin{equation*}
\begin{split}
	\int_I\psi(r)\;d\nu(r)&=\lim\limits_{k\to\infty}\int_I f_k(r)\;d\nu(r)\\
	&\leq \lim\limits_{k\to\infty}\int_I f_k(r)\;d\mu^*(r)=\int_I\psi(r)\;d\mu^*(r).
 \end{split}
\end{equation*}
	\item Si $\psi$ es cualquier función integrable, entonces se puede descomponer como diferencia de dos funciones positivas, es decir $\psi=\psi^+-\psi^-$  y por el ítem anterior satisface  que 
$$\int_I\psi(r)\;d\nu(r)\leq \int_I\psi(r)\;d\mu^*.$$	
\end{itemize}
Luego, si se verifica para cualquier intervalo  abierto $I$ entonces,  como  este tipo de conjuntos generan la $\sigma$-álgebra de Borel, se satisface para cualquier conjunto boreliano $A$.


\end{proof} 








\begin{defi}
	Sea $\mu$ una medida con signo. Llamaremos $D_{\mu}$ o simplemente $D$, al conjunto de los puntos de discontinuidad \index{punto de discontinidad} de $\mu$, es decir, 
	$$D=\{t \in [0,T]  \mid  \mu(\{t\})\neq 0\}.$$\index[Simbolo]{$D$}
\end{defi}


\begin{lem}\label{D numerable}
	Si $\mu:\mathscr{B}([0,T])\to \rr$ es una medida finita y positiva, entonces el conjunto $D$ es numerable.
\end{lem}
\begin{proof}[Dem.]
	Para $n\in\nn$ definimos el conjunto $D_n=\{\tau\mid\mu(\{\tau\})>1/n\}$, entonces $D=\bigcup_{n=1}^\infty D_n$. Cada conjunto $D_n$ es finito, porque de lo contrario va a existir una sucesión de elementos $\{a_k\}_{k=1}^\infty\subset D_n$ tal que 
	\begin{equation*}
		\mu([0,T])\geq\mu(D_n)\geq\sum_{k=1}^{\infty}\mu(\{a_k\})>\sum_{k=1}^{\infty}\dfrac{1}{n}=\infty,
	\end{equation*}
	lo cual es un absurdo, pues $\mu$ es una medida finita. Por lo tanto, como $D_n$ es finito entonces $D$ es la unión numerable de conjuntos finitos. 
\end{proof}
